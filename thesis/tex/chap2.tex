\chapter{Background}

\section{Venture Capital}

\subsection{Definitions}

For the purposes of this thesis, we will consider the standard structure of a seed-stage venture firm, as follows. A venture firm, or VC, is composed of a central pool of capital, contributed by individuals or organizations known as Limited Partners (LPs). This pool is managed by individuals known as General Partners (GPs), who are compensated for their work both with a fraction of the pool (the management fee) as well as a fraction of the returns on their investments (the carry).

In our simplified model, the sole goal of a VC is to trade capital from the pool for equity in companies that will later either enter public markets (via an Initial Public Offering, or IPO) or get acquired by another company. These liquidation events allow the VC to sell their equity for more than the original purchase price. Thus, the success of a VC is measured by the realized capital gains that are accrued when they sell equity in a now-public investment, and the objective function they optimize for is the expected value of this gain over all their investments. We will define efficiency as the fraction of hours in a given period of time the GPs must spend working in order to achieve some level of expected returns.

For more background on venture capital and the ongoing economic research in the field, we refer the reader to \cite{venture-survey}.

\subsection{Automation Opportunities}

In order to motivate our chosen problem of better matching between founders and investors, we first explore the set of opportunities for automation in venture capital as a whole, identifying our selected opportunity as particularly impactful.

The time GPs spend working is split between the activities of sourcing, analyzing, and supporting startups. One can imagine these forming a funnel-like pipeline: sourcing looks to fill the top of the funnel with as many high-quality companies as possible, analyzing seeks to filter these companies down to only the investment-worthy ones, and supporting aims to lengthen the lifespan of the existing companies by as much as possible.

While it is clear how sourcing additional companies and doing a better job of analyzing potential investments is beneficial to the bottom line of a firm, it is not self-evident that investing time into supporting portfolio companies leads to greater expected returns. To mitigate these concerns, we refer the reader to \cite{JOFI:JOFI12370}, which shows that supporting portfolio companies results in ``an increase in innovation and the likelihood of a successful exit.''.

While the proportion of time spent on each activity varies greatly between firms, we will assume a roughly equal split, such that increasing efficiency in any one is equally impactful. Importantly, these three activities correspond to the three opportunities we will consider for increased efficiency in venture capital.

\subsubsection{Sourcing}

Sourcing entails GPs leveraging their networks and any available information (free or proprietary) to discover the optimal set of companies to consider investment in. The stream of companies that are being considered are known as ``deal flow''. This is commonly split into outbound and inbound flow. Outbound flow is generated by the partners attending events and scouring their digital and analog networks for new companies being started. Inbound flow, on the other hand, is generated by startup founders reaching out to the firm and requesting consideration for investment.

Much of sourcing requires humans integrating large swaths of linked information, resulting in a few highlights in the form of ``interesting'' companies. The more data that can be ingested, the more interesting companies are surfaced. We model this as an unsupervised graph problem, where nodes represent information accessible to a firm, and explore how we can learn to identify interesting nodes at a scale no human could manage.

\subsubsection{Analyzing}

The process of analyzing and doing due diligence on startups is how the GPs of a firm decide whether or not to invest. This can include reviewing the product, financials, and traction of the startup, in addition to doing research on the founders and broader industry at hand.

The lowest-hanging fruit in this process is the notion of automatically filtering, categorizing, and ranking pipeline items. Investors currently limit the number of companies they are considering at any given time to the few they can learn absolutely everything about. Furthermore, they pass on many companies on the basis of cheap filters and pattern-matching historic successes. The problem of clustering and raking companies can be modeled as a supervised, structured problem, leveraging both historic successes as well as past misses.

\subsubsection{Supporting}

Providing what is known as ``portfolio support'' is how venture firms attempt to ensure the companies they invest in survive long enough to IPO (thereby allowing the VCs to cash out). This encompasses everything from advising the founders, to making key introductions, to helping the company raise further funds, to helping publicize big product announcements.

The opportunities in this area are focused on how we can use commoditized machine learning techniques to solve problems more optimally, and how we can build better interfaces to the existing data. This includes superior matching of founders to investors (which can help portfolio companies raise their next round expeditiously) and providing platforms or tools which automate common tasks shared across similarly-staged companies.

\section{Matching Founders to Investors}

To evaluate the above opportunities, we first lay out our criteria. The goal is to save the greatest number of hours possible, for the largest set of people in the startup ecosystem. We wish to build a tool or product which is open, free, and accessible. It should allow founders or investors anywhere to benefit, and should not be tied to a specific institution. It should become more useful as it is adopted. Finally, it should play a part in making venture funding more equitable.

There has been a great deal of attention recently on the issue of diversity and equality in the venture captial world, with many studies concluding that groups such as women and under-represented minorities are less likely to succeed in fundraising due to biases against them. For example, ``women are outside the formal, predominately male venture capital network. Consequently, it may be harder for female entrepreneurs to make the connection, to get in the door, or gain attention for their deal.''\cite{doi:10.1080/13691060118175} In building this tool, we strived to use technology to level the playing field that currently unfairly discriminates. We were careful not to build models which simply codify the existing biases in the ecosystem.

From the opportunities identified above, we proposed several potential products or tools to be build. A comprehensive overview of the proposals can be found in Appendix \ref{intro:products}.

From these, we selected the problem of efficiently matching founders to investors. Given our desired constraints of building an accessible, public platform, we posit that this opportunity is far more appropriate than any other. Many of the opportunities in Sourcing and Analyzing require a proprietary product which is kept private the the venture firm which built it. The matching problem, however, necessarily needs to be a public platform to function, as it needs to contain accurate information about the entire venture ecosystem. It incentivizes shared use, while still compensating the party that builds it by providng proprietary aggregate insights, and brand value. Building a public platform lets us se existing non-commercial data, which is invaluable given that venture firms are highly averse to sharing their private data with others (as it is what gives them a competitive advantage). Finally, building a public product that influences every stakeholder on both sides of the transaction is the simplest way to impact the largest number of people.

The problem of matching founders to investors in an efficient manner is crucial to the health of the venture ecosystem. Aside from the extremely well-known firms, there are countless venture firms in the country which are willing to invest in various niches and demographics, and often the pain of seeking out these firms is what prevents a startup from raising money as expeditiously as possible. On the flip side, venture firms are always seeking out knowledge about new companies being started, and often struggle to have to opportunity to evaluate many companies, given the competitive nature of venture capital. Indeed, ``better‐networked VC firms experience significantly better fund performance''\cite{doi:10.1111/j.1540-6261.2007.01207.x}, meaning this competition to see new deals is crucial to a firm's success.

\section{Original Proposal}

The original product we set out to build for this thesis was a recommender system for early founders to find investors, which powered a public-facing tool that collects data from founders and investors, and surfaces recommendations for each. The idea was to train this recommender system on public funding data, characterizing both founders and investors with features extracted from their companies. We would then augment these features with click and attention-based data from our public tool.

The work of Stone at UCL is the best starting point for related previous work in the area of recommendation systems for venture capital. In \cite{Stone:2013:EST:2541167.2507882}, the authors explore the difficult task of building a top-N recommendation system for venture firms considering investments - the inverse of the problem we were trying to solve. While not the same problem, Stone et al. discovered the difficulty in building a recommendation system with hyper-sparse data sets such as the set of venture fundings in the US, which is roughly the same data set we will be using (albeit from different sources). Their insight of leveraging both content-based and collaborative filtering, combined via a linear ensemble method, would have been the inspiration for our hybrid classifier.

Our initial experiments with building recommender systems in the opposite direction were not fruitful. The data was simply too sparse to render any meaningful recommendations. Rather than pursuing this line of experimentation, we set out to establish the most viable solution to the problem we set out to solve. As we detail later, it turns out that a rule-based system with a custom sorting function can provide sufficiently appropriate recommendations for the majority of founders. Thus, we abandoned building the original recommender system, and focused on our algorithms for sorting and matching in the public tool.

\section{A Tool and a Study}

This thesis focuses on VCWiz, the public tool that was built to solve the needs of founders and investors discussed above. This tool is now live at \url{https://vcwiz.co}, and aspires to be a comprehensive application for all the discovery, research, and outreach needs of a first-time founder. We will describe the design and implementation of VCWiz, which includes a graph-based interface which allows founders to explore their connections to investors. Through exposing this inferface, we obtained email-based social graph data from over 500 founders actively raising their initial rounds of funding. The remainder of this thesis details the characteristics of these founders, learnings from analyzing their fundraising, and the results of quantitative experiments on the aggregate graph. Every founder included in the study gave consent to have their data used for anonymous, aggregate purposes.

\section{Related Work}

\textbf{TODO talk about the other papers we reference?}

When it comes to products which solve for the efficiency of founder-investor matching, there are many existing solutions. These products are detailed in Section \ref{vcwiz:existing}. Our work stands out as unique among them as the only public product which has been built as the result of an academic exploration. There have also been private academic efforts to better discover and match with founders on the investor side, such as those undertaken by SignalFire\footnote{http://www.signalfire.com/} (``a  mini proprietary Google''\footnote{https://techcrunch.com/2015/10/22/watch-out-vcs-chris-farmer-says-hes-about-to-massively-disrupt-the-industry/} for venture capital), or Correlation Ventures\footnote{http://correlationvc.com} (``one of the world’s most complete databases of venture capital financings''~\footnote{http://correlationvc.com/approach/about} for use in predictive models).

There has been significant work done in financial and economic academic explorations of modeling venture capital. For example, we plan on borrowing several learnings from \cite{2017arXiv170604229H}, including the list of sector names to use as features for a company and the calculated features for both investors and founders. Another example are the various economic models summarized in \cite{venture-survey}, which include the problems of picking startups, matching founders to investors, and the interactions between venture firms and companies. While Rin et al. do not consider the practical ways we can improve these processes, they provide a background for the challenges at hand, and present mathematical abstractions which are useful in our models.

On the topic of recommending investors to founders (and vice-versa), recent years have rendered a few studies, most involving the aforementioned Thomas Stone, whose thesis on Computational Analytics for Venture Finance~\cite{stone2014computational} delves into many of the problems with the publicly-available datasets.
