\chapter{Background}
\section{Our view of the world}

\subsection{Definitions}

For the purposes of this thesis, we will consider the standard structure of a seed-stage venture firm, as follows. A venture firm, or VC, is composed of a central pool of capital, contributed by individuals or organizations known as Limited Partners (LPs). This pool is managed by individuals known as General Partners (GPs), who are compensated for their work both with a fraction of the pool (the management fee) as well as a fraction of the returns on their investments (the carry).

In our simplified model, the sole goal of a VC is to trade capital from the pool for equity in companies that will later either enter public markets (via an Initial Public Offering, or IPO) or get acquired by another company. These liquidation events allow the VC to sell their equity for more than the original purchase price. Thus, the success of a VC is measured by the realized capital gains that are accrued when they sell equity in a now-public investment, and the objective function they optimize for is the expected value of this gain over all their investments. We will define efficiency as the fraction of hours in a given period of time the GPs must spend working in order to achieve some level of expected returns.

For more background on venture capital and the ongoing economic research in the field, we refer the reader to \cite{venture-survey}.

\subsection{Opportunities}

The time GPs spend working is split between the activities of sourcing, analyzing, and supporting startups. One can imagine these forming a funnel-like pipeline: sourcing looks to fill the top of the funnel with as many high-quality companies as possible, analyzing seeks to filter these companies down to only the investment-worthy ones, and supporting aims to lengthen the lifespan of the existing companies by as much as possible.

While it is clear how sourcing additional companies and doing a better job of analyzing potential investments is beneficial to the bottom line of a firm, it is not self-evident that investing time into supporting portfolio companies leads to greater expected returns. To mitigate these concerns, we refer the reader to \cite{JOFI:JOFI12370}, which shows that supporting portfolio companies results in ``an increase in innovation and the likelihood of a successful exit.''.

% Image?

While the proportion of time spent on each activity varies greatly between firms, we will assume a roughly equal split, such that increasing efficiency in any one is equally impactful. Importantly, these three activities correspond to the three opportunities we will consider for increased efficiency in venture capital.

\subsubsection{Sourcing}

Sourcing entails GPs leveraging their networks and any available information (free or proprietary) to discover the optimal set of companies to consider investment in. The stream of companies that are being considered are known as ``deal flow''. This is commonly split into outbound and inbound flow. Outbound flow is generated by the partners attending events and scouring their digital and analog networks for new companies being started. Inbound flow, on the other hand, is generated by startup founders reaching out to the firm and requesting consideration for investment.

Much of sourcing requires humans integrating large swaths of linked information, resulting in a few highlights in the form of ``interesting'' companies. The more data that can be ingested, the more interesting companies are surfaced. We model this as an unsupervised graph problem, where nodes represent information accessible to a firm, and explore how we can learn to identify interesting nodes at a scale no human could manage.

\subsubsection{Analyzing}

The process of analyzing and doing due diligence on startups is how the GPs of a firm decide whether or not to invest. This can include reviewing the product, financials, and traction of the startup, in addition to doing research on the founders and broader industry at hand.

The lowest-hanging fruit in this process is the notion of automatically filtering, categorizing, and ranking pipeline items. Investors currently limit the number of companies they are considering at any given time to the few they can learn absolutely everything about. Furthermore, they pass on many companies on the basis of cheap filters and pattern-matching historic successes. The problem of clustering and raking companies can be modeled as a supervised, structured problem, leveraging both historic successes as well as past misses.

\subsubsection{Supporting}

Providing what is known as ``portfolio support'' is how venture firms attempt to ensure the companies they invest in survive long enough to IPO (thereby allowing the VCs to cash out). This encompasses everything from advising the founders, to making key introductions, to helping the company raise further funds, to helping publicize big product announcements.

The opportunities in this area seem to be around how we can use commoditized machine learning techniques to solve problems more optimally. These include matching founders to investors (effectively the Netflix Prize~\cite{netflixpize} problem) and predicting success of social media blasts (with something like SEISMIC~\cite{seismic}).

\section{why this is important}

\textbf{TODO why this is important}

\section{What we originally wanted to do}

\begin{quotation}
We enumerate several opportunities for software development and artificial intelligence to be applied to the day-to-day operations of seed-stage venture funds, and identify two such opportunities to build automated tools around. The first tool is a graph-based ranking system that ingests information and structure from the Internet, and outputs interesting (given a venture context) nodes. The second tool is a recommender system for early founders to find investors, alongside a public-facing tool to collect data and surface recommendations.
\end{quotation}

\subsection{Old Introduction / Opportunities}

Our goal is to successfully apply techniques from software development and artificial intelligence to the day-to-day operations of a venture capital firm, in order to increase the efficiency with which the firm deploys its capital to the optimal set of startup companies. In order to describe our path to this goal, it is first necessary to define what a venture capital firm is, what its goals are, how efficiency is defined, and how success is measured. Following this, we will explore several opportunities for modern computer science to enable venture firms to operate more efficiently, proposing a product or tool for each. Finally, we hope to build a system which implements two of these tools, and evaluate the efficacy of these tools in the real world.

We have spent time exploring possible tools that could be built to aid in various stages of the venture pipeline. Each tool is identified below, along with the motivation and a brief summary of the technical challenges involved.

\subsection{Sourcing}

We have identified two opportunities to do with sourcing, both on the outbound flow side.

\subsubsection{A system to aggregate signals from founders and predict the intent to start a company}

Founders often emit signals that indicate they are starting a new company, often long before they officially announce their new endeavor. These signals can be explicit (changing a job title on LinkedIn, or biography line on Twitter) or implicit (leaving a job, moving cities, or attending entrepreneurial events). In isolation, these signals are not strong, but in aggregate they can be strongly correlated with the intent to start a company.

We see an opportunity for a system that monitors the social networks of a GP, identifying and aggregating potential signals. The technical challenges include linking seemingly-unrelated signals across networks and schemas, and inventing a ranking algorithm which can present the most likely potential founders given a set of signals. We would likely use these signals as machine learning features.

\subsubsection{A system to discover and monitor promising out-of-network individuals and organizations}

While there are a plethora of announcements and releases online which would indicate an investment-worth company has formed, humans are not capable of monitoring and filtering the wealth of information generated on the internet on an ongoing basis. Thus, a GP's sourcing abilities are largely limited to the founders they can discover in their network.

We propose a system which treats the relevant information on the Internet as a connected, directed graph, which can be monitored and have its nodes ranked (as PageRank does for search engines). Every interesting community (such as educational institutions) could have its own independent graph, and the top-ranked nodes of each graph could be surfaced for easy human review. The technical challenges around this system include a lack of labeled training data (what constitutes an ``interesting'' node?) and the noisiness of the web (there are many sites linked from a community that contain irrelevant or even misleading information).

\subsection{Analyzing}

When is comes to analyzing, there are two major project proposal we considered.

\subsubsection{A system to filter, categorize, and rank the companies in a venture pipeline}

Many seed-stage funds suffer today from an overwhelming pipeline of startup companies to consider. There is considerable data available on these companies which seems to be correlated to how investment-worthy the company is at first glance. At the very least, the cheap filters applied by investors are mimicable through existing data (alma maters of founders, size of initial market, sentiment of partners after first meeting).

We propose a system which uses the information associated with pipeline companies to categorize each company into buckets that predict how far in the pipeline the company will move, using these buckets to filter and prioritize the pipeline. This will be an online, semi-supervised clustering problem which receives constant feedback from partners. The technical challenges include identifying and extracting the relevant features (which may include leveraging NLP techniques on descriptions, pitch decks, and meeting notes), and finding a way to incorporate user feedback in a meaningful way. Evaluation methods are also difficult to formulate a priori.

\subsubsection{A system to surface and summarize key trends and news in a given industry}

Many hours of time is wasted at venture firms serially researching and identifying key facts and risks about both a company and its broader industry. This act of information extraction and summarization is well-suited for classic Natural Language Processing.

We propose a system which ingests both internal data on the company at hand, as well as recent news and evergreen data sources (such as Wikipedia) and delivers a digest of key risks identified in the company (based on pitch decks and partner notes), as well as a one-pager on the given industry.

\subsection{Supporting}

Finally, with regards to portfolio support, there are two tools we considered building.

\subsubsection{A system for the discovery of and supporting outreach to the optimal set of seed-stage investors}

It is widely accepted in the venture industry that there is significant merit to a founder finding the ``right'' set of investors when raising money. Not only does the strategic focus of a firm and its network impact said firm's ability to help a company, but the particular focus of a partner within a firm can also influence whether or not a company even gets funded. There is strong empirical evidence that partners at venture firms do indeed specialize and focus on a very specific subset of companies~\cite{Stone:2013:EST:2541167.2507882}.

Matching a founder to the most relevant partner at each firm, and the most realistic and appropriate firms at each funding stage, is a challenging problem for humans to tackle alone.

We propose a hybrid recommender system which suggests relevant and strategic investors to founders, based on their company and ideal investor profile. This would follow the models laid out in recent literature on recommender systems~\cite{Burke2002}.

Our tool would also provide an interface for planning and tracking the process of reaching out to these investors, as a way to collect structured training data for future iterations. Technical challenges here include building a sufficiently strong user experience so as to inspire trust in the tool, determining how to identify a user as features, and building a labeled database of investors and the founders they have backed.

\subsubsection{A system to predict and propagate viral company news}

As the number of companies in a seed-stage venture firm's portfolio grows, it becomes increasingly difficult for partners to keep track of the movements of each company. This makes it difficult to identify when a company is in the process of making a big press release (which the VC could support). Furthermore, there is no easy way for a VC to know the latest public change in each of their companies.

We propose a tool to monitor the social media accounts of portfolio companies, summarizing news and sharing the posts that are estimated to be the most popular or viral. Text summarization is an open research problem that has several standardized solutions~\cite{textsummarization}, each of which can be tuned for the domain with manual feature engineering and additional rule-based systems. Estimating social media popularity and virality can be done with linear point-process models such as SEISMIC~\cite{seismic}, or more complex Bayesian models like the one presented in \cite{bayesiantweets}, which uses more features from the graph generated by the post and its shares. The biggest technical challenges here are around coaxing and tuning these algorithms to give sufficiently good results for our domain.

\section{What we actually did}

\textbf{TODO overview of the rest of the thesis}

\section{Related work}

\subsection{Investor reccomendations}

\textbf{TODO clean this up}

The work of Stone at UCL is the best starting point for related previous work in the area of recommendation systems for venture capital. In \cite{Stone:2013:EST:2541167.2507882}, the authors explore the difficult task of building a top-N recommendation system for venture firms considering investments - the inverse of the problem we are trying to solve. While not the same problem, Stone et al. discovered the difficulty in building a recommendation system with hyper-sparse data sets such as the set of venture fundings in the US, which is roughly the same data set we will be using (albeit from different sources). Their insight of leveraging both content-based and collaborative filtering, combined via a linear ensemble method, will be the inspiration for our hybrid classifier.

Current literature in recommendation systems defines (at least) two broad categories of systems: content-based and collaborative filtering~\cite{Burke2002}. Content-based systems characterize users with features extracted from the items they have preferences for, then use these features to find other items with similar features. Collaborative filtering, on the other hand, characterizes users by the set of preferences they have for a canonical set of items (without knowledge of the actual items), and suggests new items by finding users with similar preferences, and returning the items that they prefer. These two systems take different information into account, and can be combined into hybrid models that capture both perspectives.

Combining the models helps us mitigate some of the adverse effects of using one or the other. For example, content-based systems struggle to recommend items which are not associated with existing user items, since there are no similar features. Collaborative systems fix this by pulling items from similar users, agnostic of the item itself, but suffer from other issues, such as degraded results when users each only review a few items.

While there are other recommendation models to explore, it's accepted that ``learning-based technologies work best for dedicated users who are willing to invest some time making their preferences known to the system''~\cite{Burke2002}, which reflects our situation.

\subsection{Founder node ranking}

\textbf{TODO clean this up}

When it comes to ranking nodes in a graph, there is a large body of literature to reference. The canonical starting point is of course PageRank~\cite{page1999pagerank}, which recursively estimates node importance by analyzing the importance of nodes which link to the node in question. HITS~\cite{kleinberg1999authoritative}, a simple algorithm which calculates authority and hub scores, is also commonly used. Several variations exist, included a Weighted Page Rank~\cite{xing2004weighted}. While these algorithms do a good job ranking nodes in a graph for search relevance, they don't necessarily capture desirable characteristics for interesting nodes in a venture context. It is unclear if there even exists a canonical ranking for nodes in a graph for venture, since desirable properties depend on the company.

Thus, we look at learned graph node ranking algorithms. Strategies such as the Graph Neural Network (GNN)~\cite{scarselli2009graph} help neural nets directly process graphs, which has lead to the development of systems which use GNNs to rank web pages~\cite{scarselli2005graph}. Crucially, this algorithm does not require explicitly determining which factors are important for ranking, and can be learned from a small number of training examples in the form of inequalities. Further work has then been done extending the idea of GNNs, with gating and other modern enhancements~\cite{DBLP:journals/corr/LiTBZ15}.
