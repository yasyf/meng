\section{Introduction}

Our goal is to successfully apply techniques from software development and artificial intelligence to the day-to-day operations of a venture capital firm, in order to increase the efficiency with which the firm deploys its capital to the optimal set of startup companies. In order to describe our path to this goal, it is first necessary to define what a venture capital firm is, what its goals are, how efficiency is defined, and how success is measured. Following this, we will explore several opportunities for modern computer science to enable venture firms to operate more efficiently, proposing a product or tool for each. Finally, we hope to build a system which implements two of these tools, and evaluate the efficacy of these tools in the real world.

\subsection{Definitions}

For the purposes of this thesis, we will consider the standard structure of a seed-stage venture firm, as follows. A venture firm, or VC, is composed of a central pool of capital, contributed by individuals or organizations known as Limited Partners (LPs). This pool is managed by individuals known as General Partners (GPs), who are compensated for their work both with a fraction of the pool (the management fee) as well as a fraction of the returns on their investments (the carry).

In our simplified model, the sole goal of a VC is to trade capital from the pool for equity in companies that will later either enter public markets (via an Initial Public Offering, or IPO) or get acquired by another company. These liquidation events allow the VC to sell their equity for more than the original purchase price. Thus, the success of a VC is measured by the realized capital gains that are accrued when they sell equity in a now-public investment, and the objective function they optimize for is the expected value of this gain over all their investments. We will define efficiency as the fraction of hours in a given period of time the GPs must spend working in order to achieve some level of expected returns.

For more background on venture capital and the ongoing economic research in the field, we refer the reader to \cite{venture-survey}.

\subsection{Opportunities}

The time GPs spend working is split between the activities of sourcing, analyzing, and supporting startups. One can imagine these forming a funnel-like pipeline: sourcing looks to fill the top of the funnel with as many high-quality companies as possible, analyzing seeks to filter these companies down to only the investment-worthy ones, and supporting aims to lengthen the lifespan of the existing companies by as much as possible.

While it is clear how sourcing additional companies and doing a better job of analyzing potential investments is beneficial to the bottom line of a firm, it is not self-evident that investing time into supporting portfolio companies leads to greater expected returns. To mitigate these concerns, we refer the reader to \cite{JOFI:JOFI12370}, which shows that supporting portfolio companies results in ``an increase in innovation and the likelihood of a successful exit.''.

% Image?

While the proportion of time spent on each activity varies greatly between firms, we will assume a roughly equal split, such that increasing efficiency in any one is equally impactful. Importantly, these three activities correspond to the three opportunities we will consider for increased efficiency in venture capital.

\subsubsection{Sourcing}

Sourcing entails GPs leveraging their networks and any available information (free or proprietary) to discover the optimal set of companies to consider investment in. The stream of companies that are being considered are known as ``deal flow''. This is commonly split into outbound and inbound flow. Outbound flow is generated by the partners attending events and scouring their digital and analog networks for new companies being started. Inbound flow, on the other hand, is generated by startup founders reaching out to the firm and requesting consideration for investment.

Much of sourcing requires humans integrating large swaths of linked information, resulting in a few highlights in the form of ``interesting'' companies. The more data that can be ingested, the more interesting companies are surfaced. We model this as an unsupervised graph problem, where nodes represent information accessible to a firm, and explore how we can learn to identify interesting nodes at a scale no human could manage.

\subsubsection{Analyzing}

The process of analyzing and doing due diligence on startups is how the GPs of a firm decide whether or not to invest. This can include reviewing the product, financials, and traction of the startup, in addition to doing research on the founders and broader industry at hand.

The lowest-hanging fruit in this process is the notion of automatically filtering, categorizing, and ranking pipeline items. Investors currently limit the number of companies they are considering at any given time to the few they can learn absolutely everything about. Furthermore, they pass on many companies on the basis of cheap filters and pattern-matching historic successes. The problem of clustering and raking companies can be modeled as a supervised, structured problem, leveraging both historic successes as well as past misses.

\subsubsection{Supporting}

Providing what is known as ``portfolio support'' is how venture firms attempt to ensure the companies they invest in survive long enough to IPO (thereby allowing the VCs to cash out). This encompasses everything from advising the founders, to making key introductions, to helping the company raise further funds, to helping publicize big product announcements.

The opportunities in this area seem to be around how we can use commoditized machine learning techniques to solve problems more optimally. These include matching founders to investors (effectively the Netflix Prize~\cite{netflixpize} problem) and predicting success of social media blasts (with something like SEISMIC~\cite{seismic}).
